\documentclass{article}
\usepackage{amsmath,url,tensor,amssymb}
\begin{document}
\section{Sources of information}

\subsection{Scope}

Could be fun to do trajectories in as much generality as a charged
particle in Kerr-Newman (spinning charged). Implementation in
mathematica, which appears to be open source:
\url{http://notizblock.yukterez.net/viewtopic.php?p=426} .
This suggests not depending too much on facts that may only hold for
the Schwarzschild spacetime, such as the ability to cover the whole
spacetime with a single chart, or the existence of several killing vectors
with corresponding conserved quantities. And note that in fact KS coordinates
do *not* cover the whole spacetime with a single chart, because they have
a coordinate singularity at theta=0 and pi.

Could also do stuff like maneuvering of rocket after infall to postpone hitting singularity
(which *can* help, contrary to popular belief -- see notes).

\subsection{MTW}

p. 826: ``Several well-behaved coordinate systems''

Box on p. 828 presents detailed calculations and motivations.

\subsection{Hawking and Ellis}

p. 155

\subsection{Wikipedia}

Nice presentation, readable notation.

\subsection{coordinate systems}

WP: $T,X$

MTW: $v,u$

H\&E: $t',x'$

$(T,X)=(v,u)=(t'/\sqrt2,x'/\sqrt2)$

H\&E's null coordinates: $(v',w')=(t'+x',t'-x')$

H\&E's Penrose diagram: $(v'',w'')=(\tan^{-1}(v'/\sqrt2),\tan^{-1}(v'/\sqrt2))$.

\section{Transformation from Schwarzschild to KS}

Throughout the following, let $m=1/2$, so the horizon is at $r=1$, i.e., $r$ is in units of Sch.~radius.

Exterior is $r>1$, regions I and III. Interior is $r<1$, regions II and IV.

\newcommand{\lambert}{\ell}

The following is notation I made up to keep the writing simple.
Define a variable $\lambert=r-1 $.
\begin{align}
  \operatorname{sc} &= \begin{cases}
\sinh, \qquad \text{exterior}\\
\cosh, \qquad \text{interior}
\end{cases} \\
  \operatorname{cs} &= \begin{cases}
\cosh, \qquad \text{exterior}\\
\sinh, \qquad \text{interior}
\end{cases} \\
  \operatorname{tc} &= \begin{cases}
\tanh, \qquad \text{exterior}\\
\coth, \qquad \text{interior}
\end{cases}\\
\sigma &= \begin{cases}
+1, \qquad \text{regions I and II, $V>0$}\\
-1, \qquad \text{regions III and IV, $V<0$, (the universe we can't reach)} 
\end{cases}
\end{align}

Then the transformation from Schwarzschild to KS is (MTW p.~827):

\begin{align}
  T &= \sigma \sqrt{|\lambert |} e^{r/2}\operatorname{sc}\frac{t}{2} \\
  X &= \sigma \sqrt{|\lambert |} e^{r/2}\operatorname{cs}\frac{t}{2} 
\end{align}

This transformation is supposed to fix coordinate singularities at the horizons where the
metric is degenerate, so it can't be one-to-one at the the horizons.

The following is nonstandard notation that I invented for convenience.
Let
\begin{equation}
  \rho = X^2-T^2 = -VW \qquad \text{see below for null coords $VW$},
\end{equation}
which is a measure of how far you are from the singularity, taking values on $(-1,\infty)$.
It's positive on the exterior, negative on the interior. Horizon is at
$\rho=0$, both singularities at $\rho=-1$. We have
\begin{equation}
  \lambert  = W(\rho/e).
\end{equation}
$W$ is the principal real branch of the Lambert W function (more info below).
Like $\rho$, $\lambert $ takes values on $(-1,\infty)$, and
the interpretation of its sign is also the same. 

The Lambert 
W function is defined
by $z=W(ze^z)$, i.e., it's the inverse of the function inside the parens. 
Maxima implementation is \texttt{lambert\_w}. Scipy has \texttt{lambertw}.
There is a GPL'd javascript implementation: \url{https://github.com/protobi/lambertw}.
High-performance C++ implementation: \url{https://github.com/DarkoVeberic/LambertW}.
Because W is defined as the inverse of a certain function, its derivative is easy
to express in terms of itself, $dW/dz=W/(z(1+W))=1/(z+e^W)$. The latter expression
looks more expensive to compute because of the exponential, but it's safer because
it doesn't misbehave at $z=0$, where the first one is an indeterminate form.
Maxima uses the equivalent (and also safe) form $e^{-W}/(1+W)$.

For large $\rho$,
$\lambert =\ln\rho-\ln(\ln\rho-1)+o(1)$.

The metric is
\begin{equation}
  ds^2 = B(dT^2-dX^2)-(1+\lambert )^2d\Omega^2,
\end{equation}
where
\begin{equation}
  B = \frac{4\lambert }{(1+\lambert )\rho} = \frac{4(r-1)}{r\rho}  = \frac{4}{r}e^{-r}.
\end{equation}
Here $(1+1/\lambert )\rho=re^r$ by the identity $e^{W(x)}=x/W(x)$. 
This type of expression will inevitably involve radial coordinates such as $r$, $\rho$, and
$\lambert $, even though KS doesn't even explicitly talk about radial coordinates; this makes
sense because the local, intrinsic properties of the spacetime only depend on the radial
coordinate (e.g., all curvature scalars are just functions of $1/r$).

I had to correct the coefficient in $B$ from 16 to 4
in order to get results from this metric to agree with results from Schwarzschild coords.
I'm darned if I can find my math mistake, but the perfect numerical agreement with
Schwarzschild on randomly chosen points is tantamount to a proof that the version with
the 4 in it is correct. This also checks with equation 6 in \url{https://arxiv.org/pdf/1401.1337.pdf} .

The inverse transformation is
\begin{align}
  r &= 1+\lambert  \\
  t &= 2\operatorname{sc}^{-1}(T|\lambert |^{-1/2}e^{-r/2}) = 2\operatorname{tc}^{-1}(T/X).
\end{align}

At the horizons, applying the inverse transformation gives $r=1$, $t=\pm\infty$. I'm not
sure if there's a nice interpretation of this, but the fact that it's not well-behaved
and one-to-one is what we expect at a place where we were removing a degeneracy of the metric.

\section{Null KS coordinates}

Let
\begin{align}
  V &= T+X\\
  W &= T-X,
\end{align}
which are the same as H\&E's $(v'/\sqrt2,w'/\sqrt2)$.
Metric is (my calculation)
\begin{equation}
  ds^2 = B dVdW-\ldots d\Omega^2.
\end{equation}
See remarks below about checking this on v,w (Penrose) coordinates.

\section{Not well suited to large r and t}

At large values of $r$ and $|t|$, the asymptotic behavior of the $T,X$ and $V,W$ coordinates
is $re^{r/2}$ and $e^{|t|/2}$. Since the Schwarzschild radius of the sun is 3.0 km, that means
that the earth's orbit has $r\sim 10^8$, so K-S coordinates are ridiculously huge, much too
large to store as floating point numbers. For a general-purpose implementation, I see two possible
approaches:

(1) ``Charts of convenience.'' Although covering the whole spacetime with one chart is kind of the point of KS,
could use a different chart at large r. See notes in separate section on isotropic coordinates.
(Possibly
Eddington-Finkelstein are well suited to the region near the horizon?)

(2) Try to do another change of coordinates to get ones that don't misbehave so horribly at
large distances, while still covering everything with one chart. Don't want to use Penrose diagram
coordinates for this, because compactification would give them poor precision at large r. For example,
I think doing $\sinh^{-1}X$ results in a coordinate that varies linearly with r for large r. The problem
is that then the components of the metric blow up like $\cosh X$, which is the exponential of an
exponential.

(3) ``Sliding chart.'' The KS coordinates obscure the Killing vector, and make it difficult
to do computations at large $|t|$, where KS coordinates get exponentially large.
Handle this by translating to t=0, and keeping track of the translation. This would be sort of like
having different charts of convenience for different eras of time (or different places on interior,
where t is spacelike). The transformation is actually pretty
simple in V,W coordinates. We store a point 
as (t,V,W), meaning the point (V,W) translated by an interval t in Schwarzschild coordinates.
Every point has a canonical form in which $|V|=|W|$, giving $t=0$. (This is a point on the X
or T axis.) To canonicalize, we take
\begin{align*}
  t &\rightarrow t+2\operatorname{tc}^{-1}[(V+W)/(V-W)] \\
  V &\rightarrow \pm\sqrt{\mp VW} = \pm\sqrt{\pm\rho} \quad \text{preserving original sign} \\
  W &\rightarrow \pm V \text{preserving the original sign}.
\end{align*}
At the step where we  canonicalize $t$, we have to manipulate $V$ and $W$, which in theory could
be impossible to express in machine floating point. However, as long as we always keep our points
\emph{close} to being canonical, this will never be a problem. When we compute things like Christoffel
symbols, we can just use the V,W part of the representation, ignoring the value of t. Could actually
delay canonicalization until $t$ exceeds some bound like $\pm5$. We want the canonicalization to be
a do-nothing for representations that are already canonicalized. This works, and, e.g., we can
always express things in terms of $(V+W)/(V-W)$ or its inverse in such a way that it doesn't equal
infinity. E.g., in region II, where $V$ and $W$ are both positive,
express this not as $\coth^{-1}[(V+W)/(V-W)]$ but as
$\tanh^{-1}[(V-W)/(V+W)]$.

Plan: Represent points in sch spacetime using three data structures for different regions of spacetime:

(a) (far) For $r>6$,
or $\rho\gtrsim 2000$, represent as Schwarzschild coordinates.

(b) (exterior, near horizon, large time) This is $1<r<6$ and $|t|>6$. Sliding chart.

(c) ( interior) For $r<1$, represent as $(V,W)$.

\section{Coordinates for Penrose diagrams}

Let
\begin{align}
  v &= f(V)\\
  w &= f(W),
\end{align}
where $f$ is some chosen such as $\tan{-1}$ (H\&E), $(2/\pi)\tan^{-1}$ (my default),
or $\tanh$ (used sometimes by Winitzki). The reason for the inverse tangent seems to be
that in the Minkowski case, it links geometrically to the maximal extension of the Einstein static universe
on a cylinder. Using my $(2/\pi)\tan^{-1}$, we have easy ways of describing stuff:
singularities at $v+w=\pm1$, horizons at $vw=0$, future null infinity at $v=1$ or $w=1$,
and past null infinity at $v$ or $w=-1$.

Letting $g={f^{-1}}'$, the metric in terms of the Penrose coordinates is (my calculation)
\begin{equation}
  ds^2 = B g(v)g(w)dvdw-\ldots d\Omega^2.
\end{equation}
If $f=\tan^{-1}$ then $g=\sec^2$,
and if $f=\tanh$ then $g(x)=1/(1-x^2)$. This makes sense because (1) the metric is never degenerate,
(2) $g$ blows up at infinity, and (3) $ds^2=0$ for fixed angular coordinates iff $dv=0$ or $dw=0$.

Some additional checks to do: 
Check factors of 2. Check that it makes sense if $f$ is the identity.
Compute Kretschmann invariant.

\section{Christoffel symbols}

The following are generated by running christoffel.mac and cutting and pasting the output into
christoffel.rb, then cleaning up by hand. Although I am currently having some doubt about a
factor of 4 in the definition of B,
changing B by a constant factor doesn't actually change these expressions, because the
coefficients that depend on that factor
are expressed in terms of B.

\begin{align}
\Gamma\indices{^V_V_V} &= (r^{-1}+r^{-2})We^{-r} \\
\Gamma\indices{^W_W_W} &= (r^{-1}+r^{-2})Ve^{-r} \\
\Gamma\indices{^\theta_V_\theta} = \Gamma\indices{^\phi_V_\phi} &= -WB/4r  \\
\Gamma\indices{^\theta_W_\theta} = \Gamma\indices{^\phi_W_\phi} &= -VB/4r  \\
%
\Gamma\indices{^V_\theta_\theta} &= -Vr/2\\
\Gamma\indices{^W_\theta_\theta} &= -Wr/2\\
\Gamma\indices{^V_\phi_\phi} &= -(Vr/2) \sin^2\theta\\
\Gamma\indices{^W_\phi_\phi} &= -(Wr/2) \sin^2\theta\\
%
\Gamma\indices{^\theta_\phi_\phi} &= -\sin\theta \cos\theta   \\
\Gamma\indices{^\phi_\theta_\phi} &= \cot\theta   
\end{align}

Need to check these by numerical sampling against the f90(...) output from maxima.

Posted these at \url{https://physics.stackexchange.com/questions/404611/christoffel-symbols-in-kruskal-szekeres-coordinates}

\section{Isotropic coordinates}

There is a nice set of coordinates called isotropic coordinates,
described in Chrusciel, The geometry of black holes,
\url{https://homepage.univie.ac.at/piotr.chrusciel/teaching/Black%20Holes/BlackHolesViennaJanuary2015.pdf} ,
p.~19. These are basically Minkowski-ish coordinates that asymptotically approach the Minkowski
metric, with the metric being simple to express in them.  What Chrusciel describes are
the cartesian-style coordinates, while MTW describes the spherical style.

WP has an article ``isotropic coordinates,'' which doesn't superifically match that
closely with this topic. 
Similar: \url{http://ion.uwinnipeg.ca/~vincent/4500.6-001/Cosmology/IsotropicCoordinates.htm} .
See \url{https://physics.stackexchange.com/q/145342} .

MTW has p. 595, ex. 23.1, and p. 840, ex. 31.7, which deal with the spherical style.

Define $\tilde{r}$ implicitly according to
\begin{equation}
  r = \tilde{r}\left(1+\frac{1}{4\tilde{r}}\right)^2.
\end{equation}
The horizon is at $\tilde{r}=1/4$. Define coordinates $(t,x_i)$, where $t$ is
just the Schwarzschild $t$ coordinate, while $x_i$ are new coordinates that Chrusciel doesn't
explicitly define through any coordinate transformation to the Schwarzschild coordinates,
but that he does relate to Sch.~through the relation $\sum x_i^2=\tilde{r}$, and they
become Minkowski at large $r$. From comparison with MTW p. 840, eq. 31.22, I'm pretty
sure the idea is that $x_i$ are just the cartesian coordinates of a point on a sphere
of radius $\tilde{r}$, at angles $\theta$ and $\phi$. The metric has
\begin{align}
  g_{ii} &= -(1+1/4\tilde{r})^4 \\
  g_tt &= \left(\frac{1-1/4\tilde{r}}{1+1/4\tilde{r}}\right)^2.
\end{align}

\section{Reissner-Nordstrom black hole (charge, no spin)}

MTW p. 840, ex. 31.8 gives the metric in Sch coords. For ``Kruskal-like coordinates,''
they give a reference to Graves and Brill (1960) and their own fig 34.4, p. 921,
which is a Penrose diagram.

\end{document}
