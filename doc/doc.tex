\documentclass{article}
\usepackage{amsmath,url,tensor,amssymb,verbatim}
\begin{document}
\section{Sources of information}

\subsection{Scope}

Could be fun to do trajectories in as much generality as a charged
particle in Kerr-Newman (spinning charged). Implementation in
mathematica, which appears to be open source:
\url{http://notizblock.yukterez.net/viewtopic.php?p=426} .
This suggests not depending too much on facts that may only hold for
the Schwarzschild spacetime, such as the ability to cover the whole
spacetime with a single chart, or the existence of several killing vectors
with corresponding conserved quantities. And note that in fact KS coordinates
do *not* cover the whole spacetime with a single chart, because they have
a coordinate singularity at theta=0 and pi.

Could also do stuff like maneuvering of rocket after infall to postpone hitting singularity
(which *can* help, contrary to popular belief -- see notes).

\subsection{MTW}

p. 826: ``Several well-behaved coordinate systems''

Box on p. 828 presents detailed calculations and motivations.

\subsection{Hawking and Ellis}

p. 155

\subsection{Wikipedia}

Nice presentation, readable notation.

\subsection{coordinate systems}

WP: $T,X$

MTW: $v,u$

H\&E: $t',x'$

$(T,X)=(v,u)=(t'/\sqrt2,x'/\sqrt2)$

H\&E's null coordinates: $(v',w')=(t'+x',t'-x')$

H\&E's Penrose diagram: $(v'',w'')=(\tan^{-1}(v'/\sqrt2),\tan^{-1}(v'/\sqrt2))$.

\section{Transformation from Schwarzschild to KS}

Throughout the following, let $m=1/2$, so the horizon is at $r=1$, i.e., $r$ is in units of Sch.~radius.

Exterior is $r>1$, regions I and III. Interior is $r<1$, regions II and IV.

\newcommand{\lambert}{\ell}

The following is notation I made up to keep the writing simple.
Define a variable $\lambert=r-1 $.
\begin{align}
  \operatorname{sc} &= \begin{cases}
\sinh, \qquad \text{exterior}\\
\cosh, \qquad \text{interior}
\end{cases} \\
  \operatorname{cs} &= \begin{cases}
\cosh, \qquad \text{exterior}\\
\sinh, \qquad \text{interior}
\end{cases} \\
  \operatorname{tc} &= \begin{cases}
\tanh, \qquad \text{exterior}\\
\coth, \qquad \text{interior}
\end{cases}\\
\sigma &= \begin{cases}
+1, \qquad \text{regions I and II, $V>0$}\\
-1, \qquad \text{regions III and IV, $V<0$, (the universe we can't reach)} 
\end{cases}
\end{align}

Then the transformation from Schwarzschild to KS is (MTW p.~827):

\begin{align}
  T &= \sigma \sqrt{|\lambert |} e^{r/2}\operatorname{sc}\frac{t}{2} \\
  X &= \sigma \sqrt{|\lambert |} e^{r/2}\operatorname{cs}\frac{t}{2} 
\end{align}

This transformation is supposed to fix coordinate singularities at the horizons where the
metric is degenerate, so it can't be one-to-one at the the horizons.

The following is nonstandard notation that I invented for convenience.
Let
\begin{equation}
  \rho = X^2-T^2 = -VW \qquad \text{see below for null coords $VW$},
\end{equation}
which is a measure of how far you are from the singularity, taking values on $(-1,\infty)$.
It's positive on the exterior, negative on the interior. Horizon is at
$\rho=0$, both singularities at $\rho=-1$. We have
\begin{equation}
  \lambert  = W(\rho/e).
\end{equation}
$W$ is the principal real branch of the Lambert W function (more info below).
Like $\rho$, $\lambert $ takes values on $(-1,\infty)$, and
the interpretation of its sign is also the same. 

The Lambert 
W function is defined
by $z=W(ze^z)$, i.e., it's the inverse of the function inside the parens. 
Maxima implementation is \texttt{lambert\_w}. Scipy has \texttt{lambertw}.
There is a GPL'd javascript implementation: \url{https://github.com/protobi/lambertw}.
High-performance C++ implementation: \url{https://github.com/DarkoVeberic/LambertW}.
Because W is defined as the inverse of a certain function, its derivative is easy
to express in terms of itself, $dW/dz=W/(z(1+W))=1/(z+e^W)$. The latter expression
looks more expensive to compute because of the exponential, but it's safer because
it doesn't misbehave at $z=0$, where the first one is an indeterminate form.
Maxima uses the equivalent (and also safe) form $e^{-W}/(1+W)$.

For large $\rho$,
$\lambert =\ln\rho-\ln(\ln\rho-1)+o(1)$.

The metric is
\begin{equation}
  ds^2 = B(dT^2-dX^2)-(1+\lambert )^2d\Omega^2,
\end{equation}
where
\begin{equation}
  B = \frac{4\lambert }{(1+\lambert )\rho} = \frac{4(r-1)}{r\rho}  = \frac{4}{r}e^{-r}.
\end{equation}
Here $(1+1/\lambert )\rho=re^r$ by the identity $e^{W(x)}=x/W(x)$. 
This type of expression will inevitably involve radial coordinates such as $r$, $\rho$, and
$\lambert $, even though KS doesn't even explicitly talk about radial coordinates; this makes
sense because the local, intrinsic properties of the spacetime only depend on the radial
coordinate (e.g., all curvature scalars are just functions of $1/r$).

I had to correct the coefficient in $B$ from 16 to 4
in order to get results from this metric to agree with results from Schwarzschild coords.
I'm darned if I can find my math mistake, but the perfect numerical agreement with
Schwarzschild on randomly chosen points is tantamount to a proof that the version with
the 4 in it is correct. This also checks with equation 6 in \url{https://arxiv.org/pdf/1401.1337.pdf} .

The inverse transformation is
\begin{align}
  r &= 1+\lambert  \\
  t &= 2\operatorname{sc}^{-1}(T|\lambert |^{-1/2}e^{-r/2}) = 2\operatorname{tc}^{-1}(T/X).
\end{align}

At the horizons, applying the inverse transformation gives $r=1$, $t=\pm\infty$. I'm not
sure if there's a nice interpretation of this, but the fact that it's not well-behaved
and one-to-one is what we expect at a place where we were removing a degeneracy of the metric.

For small radii,
the approximation $r\approx\sqrt{2(1-VW)}$ holds. 
Error in r is 20\% at $V=W=0.9$. Error in $r$ for $V=W=1-\epsilon$
shrinks like $\epsilon^{1/2}$ (relative) or $\epsilon$ (absolute).

\section{Null KS coordinates}

Let
\begin{align}
  V &= T+X\\
  W &= T-X,
\end{align}
which are the same as H\&E's $(v'/\sqrt2,w'/\sqrt2)$.
\begin{verbatim}
  W    V
   \  /
    \/
\end{verbatim}
Metric is (my calculation)
\begin{equation}
  ds^2 = B dVdW-\ldots d\Omega^2.
\end{equation}
See remarks below about checking this on v,w (Penrose) coordinates.

\section{Not well suited to large r and t}

At large values of $r$ and $|t|$, the asymptotic behavior of the $T,X$ and $V,W$ coordinates
is $re^{r/2}$ and $e^{|t|/2}$. Since the Schwarzschild radius of the sun is 3.0 km, that means
that the earth's orbit has $r\sim 10^8$, so K-S coordinates are ridiculously huge, much too
large to store as floating point numbers. For a general-purpose implementation, I see two possible
approaches:

(1) ``Charts of convenience.'' Although covering the whole spacetime with one chart is kind of the point of KS,
could use a different chart at large r. See notes in separate section on isotropic coordinates.
(Possibly
Eddington-Finkelstein are well suited to the region near the horizon?)

(2) Try to do another change of coordinates to get ones that don't misbehave so horribly at
large distances, while still covering everything with one chart. Don't want to use Penrose diagram
coordinates for this, because compactification would give them poor precision at large r. For example,
I think doing $\sinh^{-1}X$ results in a coordinate that varies linearly with r for large r. 

(3) ``Sliding chart.'' The KS coordinates obscure the Killing vector, and make it difficult
to do computations at large $|t|$, where KS coordinates get exponentially large.
Handle this by translating to t=0, and keeping track of the translation. This would be sort of like
having different charts of convenience for different eras of time (or different places on interior,
where t is spacelike). The transformation is actually pretty
simple in V,W coordinates. We store a point 
as (t,V,W), meaning the point (V,W) translated by an interval t in Schwarzschild coordinates.
Every point has a canonical form in which $|V|=|W|$, giving $t=0$. (This is a point on the X
or T axis.) To canonicalize, we take
\begin{align*}
  t &\rightarrow t+2\operatorname{tc}^{-1}[(V+W)/(V-W)] \\
  V &\rightarrow \pm\sqrt{\mp VW} = \pm\sqrt{\pm\rho} \quad \text{preserving original sign} \\
  W &\rightarrow \pm V \text{preserving the original sign}.
\end{align*}
At the step where we  canonicalize $t$, we have to manipulate $V$ and $W$, which in theory could
be impossible to express in machine floating point. However, as long as we always keep our points
\emph{close} to being canonical, this will never be a problem. When we compute things like Christoffel
symbols, we can just use the V,W part of the representation, ignoring the value of t. Could actually
delay canonicalization until $t$ exceeds some bound like $\pm5$. We want the canonicalization to be
a do-nothing for representations that are already canonicalized. This works, and, e.g., we can
always express things in terms of $(V+W)/(V-W)$ or its inverse in such a way that it doesn't equal
infinity. E.g., in region II, where $V$ and $W$ are both positive,
express this not as $\coth^{-1}[(V+W)/(V-W)]$ but as
$\tanh^{-1}[(V-W)/(V+W)]$.

Plan: Represent points in sch spacetime using three data structures for different regions of spacetime:

(a) (far) For $r>6$,
or $\rho\gtrsim 2000$, represent as Schwarzschild coordinates.

(b) (exterior, near horizon, large time) This is $1<r<6$ and $|t|>6$. Sliding chart.

(c) ( interior) For $r<1$, represent as $(V,W)$.

This is essentially what I implemented, except that I could never get the ``sliding chart''
or ``era'' mechanism to work properly and ended up ripping it out. The closest I got it to
working was in git commit 25d94127d35e942.

\section{Coordinates for Penrose diagrams}

Let
\begin{align}
  v &= f(V)\\
  w &= f(W),
\end{align}
where $f$ is some chosen such as $\tan{-1}$ (H\&E), $(2/\pi)\tan^{-1}$ (my default),
or $\tanh$ (used sometimes by Winitzki). The reason for the inverse tangent seems to be
that in the Minkowski case, it links geometrically to the maximal extension of the Einstein static universe
on a cylinder. Using my $(2/\pi)\tan^{-1}$, we have easy ways of describing stuff:
singularities at $v+w=\pm1$, horizons at $vw=0$, future null infinity at $v=1$ or $w=1$,
and past null infinity at $v$ or $w=-1$.

Letting $g={f^{-1}}'$, the metric in terms of the Penrose coordinates is (my calculation)
\begin{equation}
  ds^2 = B g(v)g(w)dvdw-\ldots d\Omega^2.
\end{equation}
If $f=\tan^{-1}$ then $g=\sec^2$,
and if $f=\tanh$ then $g(x)=1/(1-x^2)$. This makes sense because (1) the metric is never degenerate,
(2) $g$ blows up at infinity, and (3) $ds^2=0$ for fixed angular coordinates iff $dv=0$ or $dw=0$.

Some additional checks to do: 
Check factors of 2. Check that it makes sense if $f$ is the identity.
Compute Kretschmann invariant.

\section{Christoffel symbols}

The following are generated by running christoffel.mac and cutting and pasting the output into
christoffel.rb, then cleaning up by hand. Although I am currently having some doubt about a
factor of 4 in the definition of B,
changing B by a constant factor doesn't actually change these expressions, because the
coefficients that depend on that factor
are expressed in terms of B.

\begin{align}
\Gamma\indices{^V_V_V} &= (r^{-1}+r^{-2})We^{-r} \\
\Gamma\indices{^W_W_W} &= (r^{-1}+r^{-2})Ve^{-r} \\
\Gamma\indices{^\theta_V_\theta} = \Gamma\indices{^\phi_V_\phi} &= -WB/4r  \\
\Gamma\indices{^\theta_W_\theta} = \Gamma\indices{^\phi_W_\phi} &= -VB/4r  \\
%
\Gamma\indices{^V_\theta_\theta} &= -Vr/8\\
\Gamma\indices{^W_\theta_\theta} &= -Wr/8\\
\Gamma\indices{^V_\phi_\phi} &= -(Vr/8) \sin^2\theta\\
\Gamma\indices{^W_\phi_\phi} &= -(Wr/8) \sin^2\theta\\
%
\Gamma\indices{^\theta_\phi_\phi} &= -\sin\theta \cos\theta   \\
\Gamma\indices{^\phi_\theta_\phi} &= \cot\theta   
\end{align}

Part of my test suite checks
these by numerical sampling against the near-raw output from maxima.

Posted these at \url{https://physics.stackexchange.com/questions/404611/christoffel-symbols-in-kruskal-szekeres-coordinates}

\section{Isotropic coordinates}

There is a nice set of coordinates called isotropic coordinates,
described in Chrusciel, The geometry of black holes,
\url{https://homepage.univie.ac.at/piotr.chrusciel/teaching/Black%20Holes/BlackHolesViennaJanuary2015.pdf} ,
p.~19. These are basically Minkowski-ish coordinates that asymptotically approach the Minkowski
metric, with the metric being simple to express in them.  What Chrusciel describes are
the cartesian-style coordinates, while MTW describes the spherical style.

WP has an article ``isotropic coordinates,'' which doesn't superifically match that
closely with this topic. 
Similar: \url{http://ion.uwinnipeg.ca/~vincent/4500.6-001/Cosmology/IsotropicCoordinates.htm} .
See \url{https://physics.stackexchange.com/q/145342} .

MTW has p. 595, ex. 23.1, and p. 840, ex. 31.7, which deal with the spherical style.

Define $\tilde{r}$ implicitly according to
\begin{equation}
  r = \tilde{r}\left(1+\frac{1}{4\tilde{r}}\right)^2.
\end{equation}
The horizon is at $\tilde{r}=1/4$. Define coordinates $(t,x_i)$, where $t$ is
just the Schwarzschild $t$ coordinate, while $x_i$ are new coordinates that Chrusciel doesn't
explicitly define through any coordinate transformation to the Schwarzschild coordinates,
but that he does relate to Sch.~through the relation $\sum x_i^2=\tilde{r}$, and they
become Minkowski at large $r$. From comparison with MTW p. 840, eq. 31.22, I'm pretty
sure the idea is that $x_i$ are just the cartesian coordinates of a point on a sphere
of radius $\tilde{r}$, at angles $\theta$ and $\phi$. The metric has
\begin{align}
  g_{ii} &= -(1+1/4\tilde{r})^4 \\
  g_tt &= \left(\frac{1-1/4\tilde{r}}{1+1/4\tilde{r}}\right)^2.
\end{align}

\section{Geodesics that hit the singularity}

It would be nice to be able to compute these with good precision, e.g., to play with
stuff like how much you can prolong your life by using rocket engines after falling
through the horizon. Atkinson has a discussion of error estimates for Runge-Kutta on
p. 372, the idea being that you compare step size 2h with step size h. But this has
several problems for this application: (1) I think he assumes the function is lipschitz,
which isn't true at the singularity; (2) he's discussing error bounds on coordinates,
but I want error bound on the proper time when the coordinate reaches a certain value.

In Kruskal coordinates, the Christoffel symbols end up being dominated by
$\ddot{V}\approx -r^{-2}\dot{W}$ and $\ddot{W}\approx -r^{-2}\dot{V}$.

Or in S coords, the dominant C symbol gives $\ddot{r}\approx -(1/2)r^{-3}\dot{r}^2$.

The main point here is that in terms of errors and adaptation of RK,
we have a diffeq that misbehaves roughly like $y'=f(y)=y^{-3}$, so that
maybe we expect errors of order $f^{(3)}$, or $r^{-6}$. Errors in proper
time might then be of the same order, since proper time and r are related
by a power law?

For a massive particle infalling along the Kruskal T axis, the solution to the
approximate equations of motion, using small-r approx for $r(V,W)$ (see above),
is $V=W=1-(-\lambda)^{5/4}$ (check this).
(Is this right? Seems like it doesn't allow rescaling of affine parameter, which doesn't
seem possible.) This gives $r=2(-\lambda)^{5/8}$. Exponent also seems wrong, see below.
Note that this was based on the ansatz $V=W=1-(-\lambda)^p$, which is not necessarily the same
as free-falling from rest at some initial $r_\text{max}>1$ as in MTW below, so the two
results may or may not be comparable.

Or using MTW's expressions in Schwarzschild coordinates, I get
$r=12^{2/3}(-\tau)^{2/3}$ (or throw in a factor of $(2m)^{1/3}$ for
standard units). This makes sense because the 2/3 is the same exponent
as in Kepler's laws, should be universal based on units. This is for free-falling
from rest at an initial position $r_\text{max}$, and it ends up not depending
on $r_\text{max}$.

\section{Reissner-Nordstrom black hole (charge, no spin)}

MTW p. 840, ex. 31.8 gives the metric in Sch coords. For ``Kruskal-like coordinates,''
they give a reference to Graves and Brill (1960) and their own fig 34.4, p. 921,
which is a Penrose diagram.

\section{Arcsinh coordinates}

Let
\begin{align*}
  a & = \sinh^{-1} V \\
  b & = \sinh^{-1} W. 
\end{align*}
The metric is
\begin{equation*}
  ds^2 = 2\mu dadb -\ldots,
\end{equation*}
where $\mu = (1/2)B\cosh a\cosh b$.
For large r and t, the a and b coordinates basically scale like r and t, not exponentially.
In region I ($a>0$, $b<0$), let $u$ be defined by
\begin{equation*}
  u = \ln(\rho/e) = a-b+\ln(f/4e),
\end{equation*}
where
\begin{equation*}
  f = (1-e^{-2a})(1-e^{2b})
\end{equation*}
is positive and o(1). The variable $u$ basically scales like $r$.
Then we can compute $r$ without having to directly manipulate V and W:
\begin{equation*}
  r = 1+\ell=1+W(e^u).
\end{equation*}
The function $W(e^u)$ is straightforward to compute --- see below.
The Schwarzschild time is
\begin{equation*}
  t = |a|+|b|+\ln\left(\frac{1-e^{-2|a|}}{1-e^{-2|b|}}\right).
\end{equation*}
To get the coefficient appearing in the metric,
\begin{equation*}
  \mu = \frac{(1+e^{-2a})(1+e^{2b})}{2er} e^{a-b-\ell}.
\end{equation*}
Since $\ell\approx a-b$, the quantity inside the exponential is safe to evaluate.

Computation of $W(e^u)$:
Veberic, \url{https://arxiv.org/abs/1003.1628}, sec.~2.3, shows an iterative
scheme that only requires an initial guess for $W(x)$, and that can be expressed purely
in terms of $\ln x$. (The iteration scheme actually works for negative $x$ as well, because
it uses $\ln(x/W_n)$.

For the transformation from Schwarzschild coordinates to (a,b), let
\begin{equation*}
  x_\pm = \sqrt{|r-1|}\exp\left(\frac{r\pm t}{2}\right),
\end{equation*}
which is equivalent to $x_+= V$ and $x_-=|W|$.
Then
\begin{align*}
  a &= \sinh^{-1} x_+ \\
  b &= \pm\sinh^{-1} x_- \quad \text{[minus sign in region I]} \\
\end{align*}
If $x$ is too big to be represented in floating point, then we have the approximation
$\sinh^{-1} x = \ln(2x)+O(x^{-2})$. (Depending on the floating point representation,
it could happen close to the horizon that the exponential
factor is too big to represent, but $x$ is not big enough for $x^{-2}$ to be negligible.
Therefore we may have to use additional terms in the asymptotic expansion for the arcsinh.)
The jacobian is
\begin{align*}
  \frac{\partial a}{\partial t} &= \ \ \frac{1}{2}(1+x_+^{-2})^{-1/2} \\
  \frac{\partial b}{\partial t} &= \pm\frac{1}{2}(1+x_-^{-2})^{-1/2} \qquad \text{[$+$ in I]} \\
  \frac{\partial a}{\partial r} &= \ \ (1-1/r)^{-1} \frac{\partial a}{\partial t} \\
  \frac{\partial b}{\partial r} &= -(1-1/r)^{-1} \frac{\partial b}{\partial t}.
\end{align*}
In these expressions, $x_\pm$ may be too large to evaluate, in which case the $x_\pm^{-2}$ terms
can simply be neglected. The partials with respect to $r$ blow up at the horizon. The jacobian
for the transformation $(a,b)\rightarrow(t,r)$ is found by inverting the $2\times 2$ matrix
above.

Christoffel symbols can be found by transforming from $(V,W)$ to $(a,b)$ using the
transformation law given in \url{https://math.stackexchange.com/q/248267/13618}:
\begin{align*}
\Gamma\indices{^a_a_a} &= \tanh a + (r^{-1}+r^{-2})e^{-r}\cosh a\sinh b \\
\Gamma\indices{^b_b_b} &= \tanh b + (r^{-1}+r^{-2})e^{-r}\cosh b\sinh a \\
\Gamma\indices{^\theta_a_\theta} = \Gamma\indices{^\phi_a_\phi} &= -(B/4r)\frac{\sinh b}{\cosh a}  \\
\Gamma\indices{^\theta_b_\theta} = \Gamma\indices{^\phi_b_\phi} &= -(B/4r)\frac{\sinh a}{\cosh b}  \\
%
\Gamma\indices{^a_\theta_\theta} &= -(r/8)\tanh a\\
\Gamma\indices{^b_\theta_\theta} &= -(r/8)\tanh b\\
\Gamma\indices{^a_\phi_\phi} &= -(r/8) \tanh a\sin^2\theta\\
\Gamma\indices{^b_\phi_\phi} &= -(r/8) \tanh b\sin^2\theta\\
%
\Gamma\indices{^\theta_\phi_\phi} &= -\sin\theta \cos\theta   \\
\Gamma\indices{^\phi_\theta_\phi} &= \cot\theta   
\end{align*}
I never actually use these because I use the 5-dimensional ones instead.

\section{Fictitious 5th coordinate}
Use this to avoid the need for multiple coordinate charts in order to handle the coordinate singularities
at $\theta=0$ and $\pi$. Define cartesian coordinates
\begin{align*}
  i &= \xi \sin\theta \cos\phi \\
  j &= \xi \sin\theta \sin\phi \\
  k &= \xi \cos\theta ,
\end{align*}
where $\xi=1$ is a fictitious coordinate, constrained to be 1.

So we have a ``big space'' B=$(a,b,i,j,k)$, which consists of the maximal extension of the Schwarzschild
spacetime embedded in a higher-dimensional space, and a ``little space'' L=$(t,r,\theta,\phi,[\xi])$,
consisting of regions I and II without the horizons.

The coordinate transformation $T:L\to B$ is known in closed form from
books, uses only elementary functions, and is special-cased for I vs
II. The transformation $T^{-1}:B\to L$ is known in closed form from
books, involves functions like $W$, may not exist (horizons, III, IV),
and can be extended by precomposing with a ``V-flip'' transformation
$V\rightarrow -V$. It's differentiable, because $W$'s derivative can
be expressed in terms of $W$. 

The metric in L is the standard Schwarzschild metric.
The metric in B can be found by applying the jacobian ${T^{-1}}'$. It's analytic, implicitly involves W,
and in simplified form I think it's
\begin{equation*}
  ds^2 = 2\mu dadb -r^2(di^2+dj^2+dk^2).
\end{equation*}
This can be tested numerically in a drop-dead-simple way by just taking two
infinitesimal displacements at a point in L, transforming to B, and comparing inner products.

Since we only work in the chart B, transformations of vectors don't have to be
efficient, can be undefined, only have to happen on input and output. To transform vectors
from L to B, use $T'$, which is super simple and well behaved wherever $T$ is defined.
To go the other way, numerically invert the $5\times 5$ matrix for $T'$ at that point,
and throw away the $\xi$ component.

The Christoffel symbols in B can be found by combining the inverse of the metric
(which is analytic and well behaved, involves W) with the derivatives of the metric
(expressible in terms of W). On top of this we add the fictitious centripetal force
represented by
\begin{equation*}
  \Gamma\indices{^i_i_i} = \Gamma\indices{^i_j_j} = \Gamma\indices{^i_k_k} = \frac{i}{\xi^2},
\end{equation*}
etc. We constrain $\xi=1$, but it's probably a good idea to leave the $\xi^{-2}$ factor in
there for numerical stability. The resulting expressions may be messy. I can probably find simplified
forms, and if so, they can be tested numerically.

If I take the Schwarzschild metric with $r^2 d\Omega^2\rightarrow r^2(i^2+j^2+k^2)$ and
compute its Christoffel symbols (schwarzschild5.mac), I get:
\begin{enumerate}
\item the usual Schwarzschild symbols involving t and r
\item symbols like $\Gamma\indices{^i_i_r} = 1/r$, analogous to the standard
                          $\Gamma\indices{^\theta_\theta_r} = \Gamma\indices{^\phi_\phi_r} = 1/r$
\item symbols like $\Gamma\indices{^r_i_i} = 1-r$, analogous to the standard
                         $\Gamma\indices{^r_\theta_\theta}=1-r$ and
                         $\Gamma\indices{^r_\phi_\phi}=(1-r)\sin^2\theta$
\end{enumerate}
There is nothing like the standard $\Gamma\indices{^\theta_\phi_\phi}$ and
$\Gamma\indices{^\phi_\theta_\phi}$, because $(i,j,k)$ are cartesian.

I have facilities in the module angular for renormalizing a point so that it lies on the
$\xi=1$ sphere, and for adjusting a vector so it's tangent to the sphere. Empirically,
if I do these at each step in Runge-Kutta, the final results are actually no better; it
still acts like a 4th-order method with almost exactly the same constant of proportionality.
However, it's a good idea to invoke these adjustments before and after Runge-Kutta to avoid
nonsense results, e.g., I confused myself by doing initial conditions with $i=1$ and
$di/d\lambda\ne=0$.

Christoffel symbols, nearly raw maxima output:
\begin{align*}
\Gamma\indices{^a_a_a} &= (\sinh(a)e^r(r-1)^2+(\cosh(a)^2\sinh(b)+2\sinh(a)e^r)(r-1)+\sinh(a)e^r+2\cosh(a)^2\sinh(b))/ \\
                       & \qquad (\cosh(a)e^r+2\cosh(a)e^r(r-1)+\cosh(a)e^r(r-1)^2)  \\
\Gamma\indices{^i_a_i} &= -(\cosh(a)\sinh(b))/(e^r+2e^r(r-1)+e^r(r-1)^2)  \\
\Gamma\indices{^j_a_j} &= -(\cosh(a)\sinh(b))/(e^r+2e^r(r-1)+e^r(r-1)^2)  \\
\Gamma\indices{^k_a_k} &= -(\cosh(a)\sinh(b))/(e^r+2e^r(r-1)+e^r(r-1)^2)  \\
\Gamma\indices{^b_b_b} &= (\sinh(b)e^r(r-1)^2+(\sinh(a)\cosh(b)^2+2\sinh(b)e^r)(r-1)+\sinh(b)e^r+2\sinh(a)\cosh(b)^2)/ \\
                       & \qquad (\cosh(b)e^r+2\cosh(b)e^r(r-1)+\cosh(b)e^r(r-1)^2)  \\
\Gamma\indices{^i_b_i} &= -(\sinh(a)\cosh(b))/(e^r+2e^r(r-1)+e^r(r-1)^2)  \\
\Gamma\indices{^j_b_j} &= -(\sinh(a)\cosh(b))/(e^r+2e^r(r-1)+e^r(r-1)^2)  \\
\Gamma\indices{^k_b_k} &= -(\sinh(a)\cosh(b))/(e^r+2e^r(r-1)+e^r(r-1)^2)  \\
\Gamma\indices{^a_i_i} &= -(\sinh(a)(r-1)+\sinh(a))/(2\cosh(a))  \\
\Gamma\indices{^b_i_i} &= -(\sinh(b)(r-1)+\sinh(b))/(2\cosh(b))  \\
\Gamma\indices{^a_j_j} &= -(\sinh(a)(r-1)+\sinh(a))/(2\cosh(a))  \\
\Gamma\indices{^b_j_j} &= -(\sinh(b)(r-1)+\sinh(b))/(2\cosh(b))  \\
\Gamma\indices{^a_k_k} &= -(\sinh(a)(r-1)+\sinh(a))/(2\cosh(a))  \\
\Gamma\indices{^b_k_k} &= -(\sinh(b)(r-1)+\sinh(b))/(2\cosh(b))  \\
\end{align*}
These are all analytic functions, so they should be valid in all four regions. Also, we expect that
under the transformation $(a,b)\rightarrow(-a,-b)$, each $\Gamma$ should flip signs.\footnote{because it
should change by a factor of $(-1)^n$, where $n$ is the number of indices that are $a$ or $b$}

Simplified by hand:
\begin{align*}
\Gamma\indices{^a_a_a} &= \tanh a + (r^{-1}+r^{-2})e^{-r}\cosh a\sinh b \\
\Gamma\indices{^b_b_b} &= \tanh b + (r^{-1}+r^{-2})e^{-r}\cosh b\sinh a \\
\Gamma\indices{^i_a_i} &= -r^{-2}e^{-r}\cosh a\sinh b  \\
\Gamma\indices{^i_b_i} &= -r^{-2}e^{-r}\cosh b\sinh a  \\
\Gamma\indices{^a_i_i} &= -\frac{1}{2}r\tanh a\\
\Gamma\indices{^b_i_i} &= -\frac{1}{2}r\tanh b
\end{align*}
Massaged to avoid overflows:
\begin{align*}
\Gamma\indices{^a_a_a} &= \tanh a + \frac{\mu}{2}(1+r^{-1})\tanh b \\
\Gamma\indices{^b_b_b} &= \tanh b + \frac{\mu}{2}(1+r^{-1})\tanh a \\
\Gamma\indices{^i_a_i} &= -\frac{\mu}{2}r^{-1}\tanh b  \\
\Gamma\indices{^i_b_i} &= -\frac{\mu}{2}r^{-1}\tanh a  \\
\Gamma\indices{^a_i_i} &= -\frac{1}{2}r\tanh a\\
\Gamma\indices{^b_i_i} &= -\frac{1}{2}r\tanh b
\end{align*}

\section{Incomplete geodesics}
The singularities are at $\sinh a\sinh b=1$, which unfortunately isn't as simple to express in the
$(a,b)$ coordinates as in Penrose coordinates. However, evaluating the Christoffel will always
involve computing $r$ anyway, so if we want to see how close to are to a singularity, we can always
use that. Accurately handling incomplete geodesics turned out to be very tricky on my first try.
I'm probably best off just using someone else's adaptive Runge-Kutta code.

\section{Choice of language for second try}
Clojure: Has the advantage of dual implementation. I hate java. There is runge-kutta, e.g.,
\url{https://github.com/mobius-eng/numerics}. Doesn't support infix arithmetic, so no.



\end{document}
